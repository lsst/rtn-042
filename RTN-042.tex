\documentclass[OPS,authoryear,toc]{lsstdoc}
\input{meta}

% Package imports go here.

% Local commands go here.

%If you want glossaries
%\input{aglossary.tex}
%\makeglossaries

\title{Running BPS on personal HTCondor at USDF}

% Optional subtitle
% \setDocSubtitle{A subtitle}

\author{%
Monika Adamow
}

\setDocRef{RTN-042}
\setDocUpstreamLocation{\url{https://github.com/lsst/rtn-042}}

\date{\vcsDate}

% Optional: name of the document's curator
% \setDocCurator{The Curator of this Document}

\setDocAbstract{%
At the SLAC/USDF site, install and configure a personal HTCondor pool and run BPS workflows with it.
}

% Change history defined here.
% Order: oldest first.
% Fields: VERSION, DATE, DESCRIPTION, OWNER NAME.
% See LPM-51 for version number policy.
\setDocChangeRecord{%
  \addtohist{1}{YYYY-MM-DD}{Unreleased.}{Monika Adamow}
}



\begin{document}

% Create the title page.
\maketitle
% Frequently for a technote we do not want a title page  uncomment this to remove the title page and changelog.
% use \mkshorttitle to remove the extra pages

% ADD CONTENT HERE
% You can also use the \input command to include several content files.

Setup the current lsstdistrib
Get condor - condor is not included with lsst stack,  so it needs to be installed locally. 
Run:

 pip3 install --user htcondor
 
 There are several htcondor configuration files the user needs to set up a personall condor.
 \begin{itemize}
 \item Setup.env
 \item glide.tar
 \item condor.tar
 \end{itemize}
They all can be downloaded from htps....


\appendix
% Include all the relevant bib files.
% https://lsst-texmf.lsst.io/lsstdoc.html#bibliographies
\section{References} \label{sec:bib}
\renewcommand{\refname}{} % Suppress default Bibliography section
\bibliography{local,lsst,lsst-dm,refs_ads,refs,books}

% Make sure lsst-texmf/bin/generateAcronyms.py is in your path
\section{Acronyms} \label{sec:acronyms}
\addtocounter{table}{-1}
\begin{longtable}{p{0.145\textwidth}p{0.8\textwidth}}\hline
\textbf{Acronym} & \textbf{Description}  \\\hline

DM & Data Management \\\hline
\end{longtable}

% If you want glossary uncomment below -- comment out the two lines above
%\printglossaries





\end{document}
