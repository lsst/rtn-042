\documentclass[OPS,authoryear,toc]{lsstdoc}
\input{meta}

% Package imports go here.

% Local commands go here.

%If you want glossaries
%\input{aglossary.tex}
%\makeglossaries

\title{Running BPS on personal HTCondor at USDF}

% Optional subtitle
% \setDocSubtitle{A subtitle}

\author{%
Monika Adamow
}

\setDocRef{RTN-042}
\setDocUpstreamLocation{\url{https://github.com/lsst/rtn-042}}

\date{\vcsDate}

% Optional: name of the document's curator
% \setDocCurator{The Curator of this Document}

\setDocAbstract{%
At the SLAC/USDF site, install and configure a personal HTCondor pool and run BPS workflows with it.
}

% Change history defined here.
% Order: oldest first.
% Fields: VERSION, DATE, DESCRIPTION, OWNER NAME.
% See LPM-51 for version number policy.
\setDocChangeRecord{%
  \addtohist{1}{YYYY-MM-DD}{Unreleased.}{Monika Adamow}
}



\begin{document}

% Create the title page.
\maketitle
% Frequently for a technote we do not want a title page  uncomment this to remove the title page and changelog.
% use \mkshorttitle to remove the extra pages

% ADD CONTENT HERE
% You can also use the \input command to include several content files.
\section{Basic setup}
 Setup the current LSST stack 

Get HTCondor - condor is not included with the LSST stack, so install it locally. 
Check if the HTCondor is there. Try to import it from python shell. If it is not there, install it with

 pip3 install --user htcondor  
 
There are several htcondor configuration files the user needs to set up a personal condor.
Download package condor.tar.gz, and unpack it on Rubin machines at SLAC.
The package can be downloaded from Jira (for now) 

Untar it and edit files that specify  username:

Untar it and edit files that specify your username:
\begin{itemize}
\item condor/condor_config: change  VAR_FOR_HOME
\item server_bootstrap.sh 
\end{itemize}

There are two scripts to get started: server_bootstrap.sh and  client_env_setup.sh
Use the first one to create catalogs, setup system variables and start personal condor.
The second one you use when your personal condor is running,  just to create catalogs (if they are not there) and to setup all paths.
For example, after logging out from a session and login back in.


\section{Condor glide in}
A script to create 12h glide-in to slurm is located here:

condor/glidein/exec.sl

The default configuration requests 1 node for 12 hours.
To create glide-in run:

sbatch <path_to>/exec.sl

Check the status of the glide in with

squeue

Once is running, execute bps submit <your_pipeline.yaml>


\end{itemize}




\appendix
% Include all the relevant bib files.
% https://lsst-texmf.lsst.io/lsstdoc.html#bibliographies
\section{References} \label{sec:bib}
\renewcommand{\refname}{} % Suppress default Bibliography section
\bibliography{local,lsst,lsst-dm,refs_ads,refs,books}

% Make sure lsst-texmf/bin/generateAcronyms.py is in your path
\section{Acronyms} \label{sec:acronyms}
\input{acronyms.tex}
% If you want glossary uncomment below -- comment out the two lines above
%\printglossaries





\end{document}
